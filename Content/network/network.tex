\begin{center}
     \Large{\underline{Netzwerkforensik}} \\
\end{center}

\section{MAC-Adresse}
Eine MAC-Adresse ist eine physikalische Adresse, die zur Adressierung von Netzwerkverkehr benutzt wird. Auch MAC-Adressen können gefälscht werden. Bei virtuellen Netzwerkkarten (wie sie z. B. in virtuellen Maschinen zum Einsatz kommen), sind MAC-Adressen frei wählbar. Eine MAC-Adresse ist 6 Byte lang.
\section{Sniffing}
Sniffing bezeihnet das Mitschneiden bzw. Analysieren von Netzwerkdatenverkehr. Dies kann im Wesentlichen entweder durch einen man-in-the-middle-Angrif erfolgen oder durch das allgemeine Mitlesen von Netzwerk-Datenverkehr (i. d. R. Ethernet oder WLAN), zu dem man physischen Zugang hat. 
\section{Tools}
\settowidth{\MyLen}{\texttt{option.2.spa}}
\begin{tabular}{@{}p{\the\MyLen}
		@{}p{\linewidth-\the\MyLen}@{}}
	\texttt{cURL} & Einfaches Programm zum Senden von Netzwerk-Requests. Unterstützte Protokolle sind unter anderem HTTP, HTTPS, FTP und FTPS.\\
	\texttt{dig} & Befehl zum Abfragen des Domain Name Systems (Alternative zu nslookup).\\
	\texttt{dsniff} & Tools zum Sniffen von Passwörtern und Analysieren von Netzwerkdatenverkehr allgemein.\\
	\texttt{Ettercap} & Tool zum Durchführen von Man-in-the-middle-Angriffen, beispielsweise mittels ARP-Spoofing.\\
	\texttt{filesnarf} & Dateisniffer für NFS-Datenverkehr. (In dsniff enthalten.)\\
	\texttt{mailsnarf} & Sniffer für Mails im Berkeley mbox format. (In dsniff enthalten.)\\
	\texttt{msgsnarf} & Sniffer für ältere bekannte Chat-Messenger (ICQ, IRC, MSN Messenger usw.)\\
	\texttt{nmap} & Etablierter Konsolen-basierter Portscanner.\\
	\texttt{OpenVAS} & Etablierter Schwachstellen-Scanner.\\
	\texttt{Scapy} & Tool zum Manipulieren von Paketen im Netzwerkverkehr.\\
	\texttt{urlsnarf} & Sniffer für HTTP-Requests. (In dsniff enthalten.)\\
	\texttt{pcap} & API für Sniffer, die von Tools wie Tcpdump, nmap usw. verwendet wird.\\
	\texttt{Tcpdump} & Bekannter und verbreiteter Paketsniffer (Kommandozeilentool).\\
	\texttt{Wireshark} & Etablierter Netzwerksniffer für Pakete verschiedener Protokolle\\
\end{tabular}
\section{ARP}
Das \enquote{Address Resolution Protokoll} wird bei IPv4 benutzt, um von einer IP-Adresse die MAC-Adresse zu ermitteln, unter der sie zu erreichen ist. Das entsprechende Äquivalent von ARP für IPv6 ist das \enquote{Neighbor Discovery Protocol} (NDP). Mittels \enquote{ARP -a} kann man beispielsweise ARP-Zuordnungen unter Windows auslesen.
\section{ARP-Spoofing}
Als ARP-Spoofing bezeichnet man das Verteilen von ARP-Paketen bei denen die Kombination aus MAC-Adresse und IP-Adresse falsch ist. Empfänger solcher ARP-Pakete mit falschen Informationen übernehmen diese Informationen in aller Regel, ohne Prüfungen anzustellen.
\section{Man-in-the-middle-Angriffe}
Bei dieser Art von Angriffen schaltet sich der Angreifer netzwerktopologisch gesehen zwischen einem Server und sein Ziel. Dies kann oft relativ einafch mit ARP-Spoofing erreicht werden. Der man-in-the-middle kann den Netzwerkverkehr vom Ziel nun mitlesen.\footnote{Dies bringt dem Angreifer nur für Netzwerkverkehr einen Vorteil, der unverschlüsselt vom Ziel gesendet/empfangen wird} Sofern der man-in-the-middle den Datenverkehr unverändert weiterleitet, merkt das Ziel in der Regel nichts von dem man-in-the-middle. Der Angreifer kann Datenverkehr auch unterdrücken oder verändert weiterleiten (z. B. für Phishing-Angriffe).

\clearpage
